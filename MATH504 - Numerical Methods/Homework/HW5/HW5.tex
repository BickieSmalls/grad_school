\documentclass[12pt]{article}
\usepackage{wasysym}
% for bold math symbols, \bm command:
\usepackage{bm}
% for real number set symbol, \mathcal{R}:
\usepackage{amsfonts}
\usepackage{color}
\newcommand{\tr}{^{\sf T}}
\newcommand{\m}[1]{{\bf{#1}}}
\newcommand{\g}[1]{\bm #1}
\newcommand{\C}[1]{{\cal {#1}}}
\topmargin -1.1in
%\textwidth 6.42in
\textwidth 7.00in
\textheight 10in
%\oddsidemargin -.075in
%\evensidemargin -.075in
\oddsidemargin -.30in
\evensidemargin -.30in
%-------------------------------------------------------------------------------
% get epsf.tex file, for encapsulated postscript files:
\input epsf
%-------------------------------------------------------------------------------
% macro for Postscript figures the easy way
% usage:  \postscript{file.ps}{scale}
% where scale is 1.0 for 100%, 0.5 for 50% reduction, etc.
%
\newcommand{\postscript}[2]
 {\setlength{\epsfxsize}{#2\hsize}
  \centerline{\epsfbox{#1}}}
\begin{document}
%----------------------------
\vspace{.2 cm}

\begin{center}
{ \bf Homework 5 :: MATH 504 ::  Due Tuesday, Oct 11th, 11:59 pm} \\[.2in]
\end{center}
Your homework submission must be a single pdf called ``LASTNAME-hw5.pdf" 
with your solutions to all theory problem to receive full credit. All answers must be typed in Latex. 

\begin{enumerate}
 \item Create an $5\times 5$ matrix $A$ using the command {hilb(5)} in Matlab,
or {scipy.linalg.hilbert(5)} in Python. Generate a random vector $x$, and compute $b=Ax$.
Add a tiny amount of noise to $b$, call it $\hat b$. Then recover $\hat x$ from $A\hat x=\hat b$. 

How accurate is the recovered solution? Why did this happen?
You don't need to provide any code or console output, just describe what you did and what you got in a few sentence. 

\item (Coding) Construct any $3 \times 3$ invertible symmetric matrix with no entry equal to $0$.  

\begin{itemize}
\item [a)] Using the function \textbf{eig} in Matlab or equivalent in other programming languages to find the dominant eigenvalue $\lambda^*_{\max}$ and its corresponding eigenvector $v^*$.
\item [b)] Use the Power Method to find the (approximate) dominant eigenvector $v^{(k)}$ and eigenvalue 
$\mu_k$ of this matrix for different stopping criteria 
\[
\frac{\| v^{(k)}-v^*\|_2}{\|v^*\|_2}\le \epsilon
\]
Record these data in the following table for given different $\epsilon$ values.

\begin{center}
\begin{tabular}{|c|c|c|c|c|}
 \hline
$\epsilon$   & iteration &    $|\mu_k-\lambda_{\max}^*|$ &  $\frac{\|v^{(k)}-v^*\|_2}{\|v^*\|_2}$  & $\frac{\|v^{(k)}-v^{(k-1)}\|_2}{\|v^{(k-1)}\|_2}$         \\ \hline
$10^{-3}$& && & \\ \hline
$10^{-6}$ & && & \\  \hline
$10^{-9}$& && & \\  \hline
\end{tabular}
\end{center}

\noindent
Note that in practice, we don't know the exact eigenvalues and eigenvectors. So the stopping criteria needs to be 
replaced by  $\frac{\|v^{(k)}-v^{(k-1)}\|_2}{\|v^{(k-1)}\|_2}<\epsilon$.  
\end{itemize}


\item (Coding) Build a connected network graph of $5$ nodes, that is, a network with $5$ pages.
Determine the highest rated web page using the page rank approach discussed in the lecture.



\end{enumerate}





\end{document}