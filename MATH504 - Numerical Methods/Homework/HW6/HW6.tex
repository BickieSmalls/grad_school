\documentclass[12pt]{article}
\usepackage{wasysym}
% for bold math symbols, \bm command:
\usepackage{bm}
% for real number set symbol, \mathcal{R}:
\usepackage{amsfonts}
\usepackage{color}
\newcommand{\tr}{^{\sf T}}
\newcommand{\m}[1]{{\bf{#1}}}
\newcommand{\g}[1]{\bm #1}
\newcommand{\C}[1]{{\cal {#1}}}
\topmargin -1.1in
%\textwidth 6.42in
\textwidth 7.00in
\textheight 10in
%\oddsidemargin -.075in
%\evensidemargin -.075in
\oddsidemargin -.30in
\evensidemargin -.30in
%-------------------------------------------------------------------------------
% get epsf.tex file, for encapsulated postscript files:
\input epsf
%-------------------------------------------------------------------------------
% macro for Postscript figures the easy way
% usage:  \postscript{file.ps}{scale}
% where scale is 1.0 for 100%, 0.5 for 50% reduction, etc.
%
\newcommand{\postscript}[2]
 {\setlength{\epsfxsize}{#2\hsize}
  \centerline{\epsfbox{#1}}}
\begin{document}
%----------------------------
\vspace{.2 cm}

\begin{center}
{ \bf Homework 6 :: MATH 504 ::  Due Tuesday, October 18th, 11:59 pm} \\[.2in]
\end{center}
Your homework submission must be a single pdf called ``LASTNAME-hw6.pdf" 
with your solutions to all theory problem to receive full credit. All answers must be typed in Latex. 

\begin{enumerate}

\item (Coding)
Apply a fixed point method to find a root of $\cos x= \sin x$ on $[0,\frac{\pi}{2}]$, by converting the equation into a fixed point equation
\[
x = g(x)=x+\cos x -\sin x
\]
given $x_0=0$.  Note that $\cos \frac{\pi}{4}=\sin \frac{\pi}{4}$, so $x^*:=\frac{\pi}{4}\approx 0.7853982$. Fill out the
following table
\begin{center}
\begin{tabular}{|c|c|c|c|c|}
 \hline
$k$  & $x_k$& $g(x_k)$& $e_k = |x_k-x^*|$& $e_k/e_{k-1}$         \\ \hline
$1$& && & \\ \hline
$10$& && & \\ \hline
$20$& && & \\ \hline
$30$& && & \\ \hline
\end{tabular}
\end{center}

\item Let $f(x)=x^6-x-1$.
\begin{itemize}
\item [a.] Use $4$ iterations of the Newton's method with $x_0=2$ to get an approximate 
root for this equation.
\item [b.] Use $4$ iterations of the Secant method with $x_0=2$ and $x_1=1$ to get an 
approximate root for this equation.
%\item [c.] Compare the results in (a) and (b).
\end{itemize}

\item Consider the equation $e^{100x} (x-2)=0$. Apply Newton's method several times with $x^0=1$.
What do you observe?


\end{enumerate}





\end{document}